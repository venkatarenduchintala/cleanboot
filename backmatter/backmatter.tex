Ich hoffe dieses Script hat dazu beigetragen, dass das Java-Wissen etwas
aufgefrischt wurde - oder dass die, die bereits seit Jahren mit Java
entwickelt haben, dennoch etwas mitnehmen konnten, und wenn es nur aus
Architektursicht eine neue Perspektive ermöglicht. Evtl. konnten auf diese
Weise auch ein paar \textit{modernere} Sprachfeatures wie die 
\texttt{records}, Stream-API oder das functional-core-Pattern mitgenommen
werden.

Gerade im Sicherheitsbereich wurden viele Dinge gar nicht thematisiert.
Das war aber ganz bewusst, da wir die Security-Belange ausführlich im
Kurs thematisieren. Das Skript sollte an dieser Stelle nicht redundant werden.
Über was wir auch noch an keinem Punkt gesprochen haben sind die verschiedenen
Deployment-Optionen. Auch das machen wir noch im Kurs.

Wer das Skript bis zum Schluss durchgelesen hat, sollte (zumindest was)
Java angeht, bestens auf den Kurs vorbereitet sein. 

\section{Contribution}
Wer zu dem Sckript etwas beitragen möchte sei herzlich dazu eingeladen.
Es wird über GitHub öffentlich als OER zur Verfügung gestellt und kann
(und soll!) gerne geforked werden. Gerne dürfen auch Issues aufgebommen
oder pull-requests gesendet werden. Für die Vorbereitung unseres Kureses
reicht der Inhalt aktuell aus meiner Sicht aber das Ganze darf sich gerne
auch zu einem eigenständigen Lehrmaterial weiterentwickeln.

Zum Abschluss wünsche ich allen Teilnehmenden viel Spass mit dem Kurs
\textit{CAS Secure Software Design \& Development} and der ZHAW.

